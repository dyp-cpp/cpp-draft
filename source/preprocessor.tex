%!TEX root = std.tex
\rSec0[cpp]{Preprocessing directives}%
\indextext{preprocessing directives|(}

\indextext{compiler control line|see{preprocessing directives}}%
\indextext{control line|see{preprocessing directives}}%
\indextext{directive, preprocessing|see{preprocessing directives}}

%gram: \rSec1[gram.cpp]{Preprocessing directives}
%gram:

\pnum
A \defn{preprocessing directive} consists of a sequence of preprocessing tokens
that satisfies the following constraints:
The first token in the sequence is a
\tcode{\#}
preprocessing token that (at the start of translation phase 4)
is either the first character in the source file
(optionally after white space containing no new-line characters)
or that follows white space containing at least one new-line character.
The last token in the sequence is the first new-line character
that follows the first token in the sequence.\footnote{Thus,
preprocessing directives are commonly called ``lines.''
These ``lines'' have no other syntactic significance,
as all white space is equivalent except in certain situations
during preprocessing (see the
\tcode{\#}
character string literal creation operator in~\ref{cpp.stringize}, for example).}
A new-line character ends the preprocessing directive even if it occurs
within what would otherwise be an invocation of a function-like macro.

\begin{bnf}
\nontermdef{preprocessing-file}\br
    group\opt
\end{bnf}

\begin{bnf}
\nontermdef{group}\br
    group-part\br
    group group-part
\end{bnf}

\begin{bnf}
\nontermdef{group-part}\br
    if-section\br
    control-line\br
    text-line\br
    \# non-directive
\end{bnf}

\begin{bnf}
\nontermdef{if-section}\br
    if-group elif-groups\opt else-group\opt endif-line
\end{bnf}

\begin{bnftab}
\nontermdef{if-group}\br
\>\terminal{\# if}\>\>constant-expression new-line group\opt\br
\>\terminal{\# ifdef}\>\>identifier new-line group\opt\br
\>\terminal{\# ifndef}\>\>identifier new-line group\opt
\end{bnftab}

\begin{bnf}
\nontermdef{elif-groups}\br
    elif-group\br
    elif-groups elif-group
\end{bnf}

\begin{bnftab}
\nontermdef{elif-group}\br
\>\terminal{\# elif}\>\>constant-expression new-line group\opt
\end{bnftab}

\begin{bnftab}
\nontermdef{else-group}\br
\>\terminal{\# else}\>\>new-line group\opt
\end{bnftab}

\begin{bnftab}
\nontermdef{endif-line}\br
\>\terminal{\# endif}\>\>new-line
\end{bnftab}

\begin{bnftab}
\nontermdef{control-line}\br
\>\terminal{\# include}\>\>pp-tokens new-line\br
\>\terminal{\# define}\>\>identifier replacement-list new-line\br
\>\terminal{\# define}\>\>identifier lparen identifier-list\opt \terminal{)} replacement-list new-line\br
\>\terminal{\# define}\>\>identifier lparen \terminal{... )} replacement-list new-line\br
\>\terminal{\# define}\>\>identifier lparen identifier-list, \terminal{... )} replacement-list new-line\br
\>\terminal{\# undef}\>\>identifier new-line\br
\>\terminal{\# line}\>\>pp-tokens new-line\br
\>\terminal{\# error}\>\>pp-tokens\opt new-line\br
\>\terminal{\# pragma}\>\>pp-tokens\opt new-line\br
\>\terminal{\# }new-line
\end{bnftab}

\begin{bnf}
\nontermdef{text-line}\br
    pp-tokens\opt{} new-line
\end{bnf}

\begin{bnf}
\nontermdef{non-directive}\br
    pp-tokens new-line
\end{bnf}

\begin{bnf}
\nontermdef{lparen}\br
    \descr{a \terminal{(} character not immediately preceded by white-space}
\end{bnf}

\begin{bnf}
\nontermdef{identifier-list}\br
    identifier\br
    identifier-list \terminal{,} identifier
\end{bnf}

\begin{bnf}
\nontermdef{replacement-list}\br
    pp-tokens\opt
\end{bnf}

\begin{bnf}
\nontermdef{pp-tokens}\br
    preprocessing-token\br
    pp-tokens preprocessing-token
\end{bnf}

\begin{bnf}
\nontermdef{new-line}\br
    \descr{the new-line character}
\end{bnf}

\pnum
A text line shall not begin with a \tcode{\#} preprocessing token.
A non-directive shall not begin with any of the directive names appearing in the
syntax.

\pnum
When in a group that is skipped~(\ref{cpp.cond}), the directive
syntax is relaxed to allow any sequence of preprocessing tokens to occur between
the directive name and the following new-line character.

\pnum
The only white-space characters that shall appear
between preprocessing tokens
within a preprocessing directive
(from just after the introducing
\tcode{\#}
preprocessing token through just before the terminating new-line character)
are space and horizontal-tab
(including spaces that have replaced comments
or possibly other white-space characters
in translation phase 3).

\pnum
The implementation can
process and skip sections of source files conditionally,
include other source files,
and replace macros.
These capabilities are called
\term{preprocessing},
because conceptually they occur
before translation of the resulting translation unit.

\pnum
The preprocessing tokens within a preprocessing directive
are not subject to macro expansion unless otherwise stated.

\enterexample In:

\begin{codeblock}
#define EMPTY
EMPTY   #   include <file.h>
\end{codeblock}

the sequence of preprocessing tokens on the second line is \emph{not}
a preprocessing directive, because it does not begin with a \# at the start of
translation phase 4, even though it will do so after the macro \tcode{EMPTY}
has been replaced.\exitexample

\rSec1[cpp.cond]{Conditional inclusion}%
\indextext{preprocessing directive!conditional inclusion}%
\indextext{inclusion!conditional|see{preprocessing directive, conditional inclusion}}

\pnum
The expression that controls conditional inclusion
shall be an integral constant expression except that
identifiers
(including those lexically identical to keywords)
are interpreted as described below\footnote{Because the controlling constant expression is evaluated
during translation phase 4,
all identifiers either are or are not macro names ---
there simply are no keywords, enumeration constants, etc.}
and it may contain unary operator expressions of the form

\begin{ncsimplebnf}
\terminal{defined} identifier
\end{ncsimplebnf}

or

\begin{ncsimplebnf}
\terminal{defined (} identifier \terminal{)}
\end{ncsimplebnf}

which evaluate to
\tcode{1}
if the identifier is currently defined
as a macro name
(that is, if it is predefined
or if it has been the subject of a
\tcode{\#define}
preprocessing directive
without an intervening
\tcode{\#undef}
directive with the same subject identifier), \tcode{0} if it is not.

\pnum
Each preprocessing token that remains (in the list of preprocessing tokens that
will become the controlling expression)
after all macro replacements have occurred
shall be in the lexical form of a token~(\ref{lex.token}).

\pnum
Preprocessing directives of the forms

\begin{ncbnftab}
\indextext{\idxcode{\#if}}%
\terminal{\# if}\>\>constant-expression new-line group\opt\br
\indextext{\idxcode{\#elif}}%
\terminal{\# elif}\>\>constant-expression new-line group\opt
\end{ncbnftab}

check whether the controlling constant expression evaluates to nonzero.

\pnum
Prior to evaluation,
macro invocations in the list of preprocessing tokens
that will become the controlling constant expression
are replaced
(except for those macro names modified by the
\tcode{defined}
unary operator),
just as in normal text.
If the token
\tcode{defined}
is generated as a result of this replacement process
or use of the
\tcode{defined}
unary operator does not match one of the two specified forms
prior to macro replacement,
the behavior is undefined.
After all replacements due to macro expansion and the
\tcode{defined}
unary operator have been performed,
all remaining identifiers and keywords\footnote{An alternative
token~(\ref{lex.digraph}) is not an identifier,
even when its spelling consists entirely of letters and underscores.
Therefore it is not subject to this replacement.},
except for
\tcode{true}
and
\tcode{false},
are replaced with the pp-number
\tcode{0},
and then each preprocessing token is converted into a token.
The resulting tokens comprise the controlling constant expression
which is evaluated according to the rules of~\ref{expr.const}
using arithmetic that has at least the ranges specified
in~\ref{support.limits}. For the purposes of this token conversion and evaluation
all signed and unsigned integer types
act as if they have the same representation as, respectively,
\tcode{intmax_t} or \tcode{uintmax_t}~(\ref{cstdint}).\footnote{Thus on an
implementation where \tcode{std::numeric_limits<int>::max()} is \tcode{0x7FFF}
and \tcode{std::numeric_limits<unsigned int>::max()} is \tcode{0xFFFF},
the integer literal \tcode{0x8000} is signed and positive within a \tcode{\#if}
expression even though it is unsigned in translation phase
7~(\ref{lex.phases}).}
This includes interpreting character literals, which may involve
converting escape sequences into execution character set members.
Whether the numeric value for these character literals
matches the value obtained when an identical character literal
occurs in an expression
(other than within a
\tcode{\#if}
or
\tcode{\#elif}
directive)
is \impldef{numeric values of character literals in \tcode{\#if}
directives}.\cprotect\footnote{Thus, the constant expression in the following
\tcode{\#if}
directive and
\tcode{if}
statement is not guaranteed to evaluate to the same value in these two
contexts.
\begin{codeblock}
#if 'z' - 'a' == 25
if ('z' - 'a' == 25)
\end{codeblock}
}
Also, whether a single-character character literal may have a negative
value is \impldef{negative value of character literal in preprocessor}.
Each subexpression with type
\tcode{bool}
is subjected to integral promotion before processing continues.

\pnum
Preprocessing directives of the forms

\begin{ncbnftab}
\terminal{\# ifdef}\>\>identifier new-line group\opt\br
\indextext{\idxcode{\#ifdef}}%
\terminal{\# ifndef}\>\>identifier new-line group\opt
\indextext{\idxcode{\#ifndef}}%
\end{ncbnftab}

check whether the identifier is or is not currently defined as a macro name.
Their conditions are equivalent to
\tcode{\#if}
\tcode{defined}
\term{identifier}
and
\tcode{\#if}
\tcode{!defined}
\term{identifier}
respectively.

\pnum
Each directive's condition is checked in order.
If it evaluates to false (zero),
the group that it controls is skipped:
directives are processed only through the name that determines
the directive in order to keep track of the level
of nested conditionals;
the rest of the directives' preprocessing tokens are ignored,
as are the other preprocessing tokens in the group.
Only the first group
whose control condition evaluates to true (nonzero) is processed.
If none of the conditions evaluates to true,
and there is a
\tcode{\#else}
\indextext{\idxcode{\#else}}%
directive,
the group controlled by the
\tcode{\#else}
is processed; lacking a
\tcode{\#else}
directive, all the groups until the
\tcode{\#endif}
\indextext{\idxcode{\#endif}}%
are skipped.\footnote{As indicated by the syntax,
a preprocessing token shall not follow a
\tcode{\#else}
or
\tcode{\#endif}
directive before the terminating new-line character.
However,
comments may appear anywhere in a source file,
including within a preprocessing directive.}

\rSec1[cpp.include]{Source file inclusion}
\indextext{preprocessing directives!header inclusion}
\indextext{preprocessing directives!source-file inclusion}
\indextext{inclusion!source~file|see{preprocessing directives, source-file inclusion}}%
\indextext{\idxcode{\#include}}%

\pnum
A
\tcode{\#include}
directive shall identify a header or source file
that can be processed by the implementation.

\pnum
\indextext{\idxcode{<...>}|see{preprocessing directive, header}}%
A preprocessing directive of the form

\begin{ncsimplebnf}
\terminal{\# include <}h-char-sequence\terminal{>} new-line
\end{ncsimplebnf}

searches a sequence of
\impldef{sequence of places searched for a header}
places
for a header identified uniquely by the specified sequence
between the
\tcode{<}
and
\tcode{>}
delimiters,
and causes the replacement of that
directive by the entire contents of the header.
How the places are specified
or the header identified
is \impldef{search locations for \tcode{<>} header}.

\pnum
\indextext{\idxcode{\"{}...\"{}}|see{preprocessing directives, source-file inclusion}}%
A preprocessing directive of the form

\begin{ncsimplebnf}
\terminal{\# include "}q-char-sequence\terminal{"} new-line
\end{ncsimplebnf}

causes the replacement of that
directive by the entire contents of the
source file identified by the specified sequence between the
\tcode{"}
delimiters.
The named source file is searched for in an
\impldef{manner of search for included source file}
manner.
If this search is not supported,
or if the search fails,
the directive is reprocessed as if it read

\begin{ncsimplebnf}
\terminal{\# include <}h-char-sequence\terminal{>} new-line
\end{ncsimplebnf}

with the identical contained sequence (including
\tcode{>}
characters, if any) from the original directive.

\pnum
A preprocessing directive of the form

\begin{ncsimplebnf}
\terminal{\# include} pp-tokens new-line
\end{ncsimplebnf}

(that does not match one of the two previous forms) is permitted.
The preprocessing tokens after
\tcode{include}
in the directive are processed just as in normal text
(i.e., each identifier currently defined as a macro name is replaced by its
replacement list of preprocessing tokens).
If the directive resulting after all replacements does not match
one of the two previous forms, the behavior is
undefined.\footnote{Note that adjacent string literals are not concatenated into
a single string literal
(see the translation phases in~\ref{lex.phases});
thus, an expansion that results in two string literals is an
invalid directive.}
The method by which a sequence of preprocessing tokens between a
\tcode{<}
and a
\tcode{>}
preprocessing token pair or a pair of
\tcode{"}
characters is combined into a single header name
preprocessing token is \impldef{search locations for \tcode{""""} header}.

\pnum
The implementation shall provide unique mappings for
sequences consisting of one or more
\grammarterm{nondigit}{s} or \grammarterm{digit}{s}~(\ref{lex.name})
followed by a period
(\tcode{.})
and a single
\grammarterm{nondigit}.
The first character shall not be a \grammarterm{digit}.
The implementation may ignore distinctions of alphabetical case.

\pnum
A
\tcode{\#include}
preprocessing directive may appear
in a source file that has been read because of a
\tcode{\#include}
directive in another file,
up to an \impldef{nesting limit for \tcode{\#include} directives} nesting limit.

\pnum
\enternote
Although an implementation may provide a mechanism for making arbitrary
source files available to the \tcode{< >} search, in general
programmers should use the \tcode{< >} form for headers provided
with the implementation, and the \tcode{" "} form for sources
outside the control of the implementation. For instance:

\begin{codeblock}
#include <stdio.h>
#include <unistd.h>
#include "usefullib.h"
#include "myprog.h"
\end{codeblock}

\exitnote

\pnum
\enterexample
This illustrates macro-replaced
\tcode{\#include}
directives:

\begin{codeblock}
#if VERSION == 1
    #define INCFILE  "vers1.h"
#elif VERSION == 2
    #define INCFILE  "vers2.h"   // and so on
#else
    #define INCFILE  "versN.h"
#endif
#include INCFILE
\end{codeblock}
\exitexample

\rSec1[cpp.replace]{Macro replacement}%
\indextext{macro!replacement|(}%
\indextext{replacement!macro|see{macro, replacement}}%
\indextext{preprocessing directives!macro replacement|see{macro, replacement}}

\pnum
\indextext{macro!replacement list}%
Two replacement lists are identical if and only if
the preprocessing tokens in both have
the same number, ordering, spelling, and white-space separation,
where all white-space separations are considered identical.

\pnum
An identifier currently defined as an
\indextext{object-like macro|see{macro, object-like}}%
\indextext{macro!object-like}%
\grammarterm{object-like}
macro may be redefined by another
\tcode{\#define}
preprocessing directive provided that the second definition is an
object-like macro definition and the two replacement lists
are identical, otherwise the program is ill-formed.
Likewise, an identifier currently defined as
a
\indextext{function-like macro|see{macro, function-like}}%
\indextext{macro!function-like}%
\grammarterm{function-like}
macro may be redefined by another
\tcode{\#define}
preprocessing directive provided that the second definition is a
function-like macro definition that has the same number and spelling
of parameters,
and the two replacement lists are identical,
otherwise the program is ill-formed.

\pnum
\indextext{macro!replacement list}%
There shall be white-space between the identifier and the replacement list
in the definition of an object-like macro.

\pnum
If the identifier-list in the macro definition does not end with
an ellipsis, the number of arguments (including those arguments consisting
of no preprocessing tokens)
in an invocation of a function-like macro shall
equal the number of parameters in the macro definition. Otherwise, there shall be more arguments in the invocation than there are
parameters in the macro definition (excluding the \tcode{...}). There
shall exist a
\tcode{)}
preprocessing token that terminates the invocation.

\pnum
\indextext{\xname{VA_ARGS}@\mname{VA_ARGS}}%
The identifier \mname{VA_ARGS} shall occur only in the replacement-list
of a function-like macro that uses the ellipsis notation in the parameters.

\pnum
A parameter identifier in a function-like macro
shall be uniquely declared within its scope.

\pnum
The identifier immediately following the
\tcode{define}
is called the
\indextext{macro!name}%
\indextext{name!macro|see{macro, name}}%
\term{macro name}.
There is one name space for macro names.
Any white-space characters preceding or following the
replacement list of preprocessing tokens are not considered
part of the replacement list for either form of macro.

\pnum
If a
\indextext{\#\#0~operator@\tcode{\#}~operator}
\tcode{\#}
preprocessing token,
followed by an identifier,
occurs lexically
at the point at which a preprocessing directive could begin,
the identifier is not subject to macro replacement.

\pnum
A preprocessing directive of the form

\begin{ncsimplebnf}
\terminal{\# define} identifier replacement-list new-line
\indextext{\idxcode{\#define}}%
\end{ncsimplebnf}

defines an
\indextext{macro!object-like}%
\grammarterm{object-like macro} that
causes each subsequent instance of the macro name\footnote{Since, by macro-replacement time,
all character literals and string literals are preprocessing tokens,
not sequences possibly containing identifier-like subsequences
(see \ref{lex.phases}, translation phases),
they are never scanned for macro names or parameters.}
to be replaced by the replacement list of preprocessing tokens
that constitute the remainder of the directive.\footnote{An alternative token~(\ref{lex.digraph}) is not an identifier,
even when its spelling consists entirely of letters and underscores.
Therefore it is not possible to define a macro
whose name is the same as that of an alternative token.}
The replacement list is then rescanned for more macro names as
specified below.

\pnum
A preprocessing directive of the form

\begin{ncsimplebnf}
\terminal{\# define} identifier lparen identifier-list\opt \terminal{)} replacement-list new-line\br
\terminal{\# define} identifier lparen \terminal{...} \terminal{)} replacement-list new-line\br
\terminal{\# define} identifier lparen identifier-list \terminal{, ...} \terminal{)} replacement-list new-line
\end{ncsimplebnf}

\indextext{macro!function-like}%
defines a \grammarterm{function-like macro}
with parameters, whose use is
similar syntactically to a function call.
The parameters
\indextext{parameters!macro}%
are specified by the optional list of identifiers,
whose scope extends from their declaration in the identifier list
until the new-line character that terminates the
\tcode{\#define}
preprocessing directive.
Each subsequent instance of the function-like macro name followed by a
\tcode{(}
as the next preprocessing token
introduces the sequence of preprocessing tokens that is replaced
by the replacement list in the definition
(an invocation of the macro).
\indextext{invocation!macro}%
The replaced sequence of preprocessing tokens is terminated by the matching
\tcode{)}
preprocessing token, skipping intervening matched pairs of left and
right parenthesis preprocessing tokens.
Within the sequence of preprocessing tokens making up an invocation
of a function-like macro,
new-line is considered a normal white-space character.

\pnum
\indextext{macro!function-like!arguments}%
The sequence of preprocessing tokens
bounded by the outside-most matching parentheses
forms the list of arguments for the function-like macro.
The individual arguments within the list
are separated by comma preprocessing tokens,
but comma preprocessing tokens between matching
inner parentheses do not separate arguments.
If there are sequences of preprocessing tokens within the list of
arguments that would otherwise act as preprocessing directives,\footnote{Despite the name, a non-directive is a preprocessing directive.}
the behavior is undefined.

\pnum
\indextext{macro!function-like!arguments}%
If there is a \tcode{...} immediately preceding the \tcode{)} in the
function-like macro
definition, then the trailing arguments, including any separating comma preprocessing
tokens, are merged to form a single item: the \term{variable arguments}. The number of
arguments so combined is such that, following merger, the number of arguments is
one more than the number of parameters in the macro definition (excluding the
\tcode{...}).

\rSec2[cpp.subst]{Argument substitution}%
\indextext{macro!argument substitution}%
\indextext{argument~substitution|see{macro, argument substitution}}%

\pnum
After the arguments for the invocation of a function-like macro have
been identified, argument substitution takes place.
A parameter in the replacement list, unless preceded by a
\tcode{\#}
or
\tcode{\#\#}
preprocessing token or followed by a
\tcode{\#\#}
preprocessing token (see below),
is replaced by the corresponding argument after all macros
contained therein have been expanded.
Before being substituted,
each argument's preprocessing tokens are completely
macro replaced as if they formed the rest of the
preprocessing file;
no other preprocessing tokens are available.

\pnum
An identifier \mname{VA_ARGS} that occurs in the replacement list
shall be treated as if it were a parameter, and the variable arguments shall form
the preprocessing tokens used to replace it.

\rSec2[cpp.stringize]{The \tcode{\#} operator}%
\indextext{\#\#0~operator@\tcode{\#}~operator}%
\indextext{stringize|see{\tcode{\#}}}

\pnum
Each
\tcode{\#}
preprocessing token in the replacement list for a function-like
macro shall be followed by a parameter as the next preprocessing
token in the replacement list.

\pnum
A \defn{character string literal} is a \grammarterm{string-literal} with no prefix.
If, in the replacement list, a parameter is immediately
preceded by a
\tcode{\#}
preprocessing token,
both are replaced by a single character string literal preprocessing token that
contains the spelling of the preprocessing token sequence for the
corresponding argument.
Each occurrence of white space between the argument's preprocessing
tokens becomes a single space character in the character string literal.
White space before the first preprocessing token and after the last
preprocessing token comprising the argument is deleted.
Otherwise, the original spelling of each preprocessing token in the
argument is retained in the character string literal,
except for special handling for producing the spelling of
string literals and character literals:
a
\tcode{\textbackslash}
character is inserted before each
\tcode{"}
and
\tcode{\textbackslash}
character of a character literal or string literal
(including the delimiting
\tcode{"}
characters).
If the replacement that results is not a valid character string literal,
the behavior is undefined. The character string literal corresponding to
an empty argument is \tcode{""}.
The order of evaluation of
\tcode{\#}
and
\tcode{\#\#}
operators is unspecified.

\rSec2[cpp.concat]{The \tcode{\#\#} operator}%
\indextext{\#\#1 operator@\tcode{\#\#} operator}%
\indextext{concatenation!macro argument|see{\tcode{\#\#}}}

\pnum
A
\tcode{\#\#}
preprocessing token shall not occur at the beginning or
at the end of a replacement list for either form
of macro definition.

\pnum
If, in the replacement list of a function-like macro, a parameter is
immediately preceded or followed by a
\tcode{\#\#}
preprocessing token, the parameter is replaced by the
corresponding argument's preprocessing token sequence; however, if an argument consists of no preprocessing tokens, the parameter is
replaced by a placemarker preprocessing token instead.\footnote{Placemarker preprocessing tokens do not appear in the syntax
because they are temporary entities that exist only within translation phase 4.}

\pnum
For both object-like and function-like macro invocations, before the
replacement list is reexamined for more macro names to replace,
each instance of a
\tcode{\#\#}
preprocessing token in the replacement list
(not from an argument) is deleted and the
preceding preprocessing token is concatenated
with the following preprocessing token.
Placemarker preprocessing tokens are handled specially: concatenation
of two placemarkers results in a single placemarker preprocessing token, and
concatenation of a placemarker with a non-placemarker preprocessing token results
in the non-placemarker preprocessing token.
If the result is not a valid preprocessing token,
the behavior is undefined.
The resulting token is available for further macro replacement.
The order of evaluation of
\tcode{\#\#}
operators is unspecified.

\enterexample In the following fragment:

\begin{codeblock}
#define hash_hash # ## #
#define mkstr(a) # a
#define in_between(a) mkstr(a)
#define join(c, d) in_between(c hash_hash d)
char p[] = join(x, y);          // equivalent to
                                // char p[] = "x \#\# y";
\end{codeblock}

The expansion produces, at various stages:

\begin{codeblock}
join(x, y)
in_between(x hash_hash y)
in_between(x ## y)
mkstr(x ## y)
"x ## y"
\end{codeblock}

In other words, expanding \tcode{hash_hash} produces a new token,
consisting of two adjacent sharp signs, but this new token is not the
\tcode{\#\#} operator. \exitexample

\rSec2[cpp.rescan]{Rescanning and further replacement}%
\indextext{macro!rescanning and replacement}%
\indextext{rescanning and replacement|see{macro, rescanning and replacement}}

\pnum
After all parameters in the replacement list have been substituted and \tcode{\#} and \tcode{\#\#} processing has taken
place, all placemarker preprocessing tokens are removed. Then
the resulting preprocessing token sequence is rescanned, along with all
subsequent preprocessing tokens of the source file, for more macro names
to replace.

\pnum
If the name of the macro being replaced is found during this scan of
the replacement list
(not including the rest of the source file's preprocessing tokens),
it is not replaced.
Furthermore,
if any nested replacements encounter the name of the macro being replaced,
it is not replaced.
These nonreplaced macro name preprocessing tokens are no longer available
for further replacement even if they are later (re)examined in contexts
in which that macro name preprocessing token would otherwise have been
replaced.

\pnum
The resulting completely macro-replaced preprocessing token sequence
is not processed as a preprocessing directive even if it resembles one,
but all pragma unary operator expressions within it are then processed as
specified in~\ref{cpp.pragma.op} below.

\rSec2[cpp.scope]{Scope of macro definitions}%
\indextext{macro!scope of definition}%
\indextext{scope!macro definition|see{macro, scope of definition}}

\pnum
A macro definition lasts
(independent of block structure)
until a corresponding
\tcode{\#undef}
directive is encountered or
(if none is encountered)
until the end of the translation unit.
Macro definitions have no significance after translation phase 4.

\pnum
A preprocessing directive of the form

\begin{ncsimplebnf}
\terminal{\# undef} identifier new-line
\indextext{\idxcode{\#undef}}%
\end{ncsimplebnf}

causes the specified identifier no longer to be defined as a macro name.
It is ignored if the specified identifier is not currently defined as
a macro name.

\pnum
\enterexample
The simplest use of this facility is to define a ``manifest constant,''
as in
\begin{codeblock}
#define TABSIZE 100
int table[TABSIZE];
\end{codeblock}
\exitexample

\pnum
\enterexample
The following defines a function-like
macro whose value is the maximum of its arguments.
It has the advantages of working for any compatible types of the arguments
and of generating in-line code without the overhead of function calling.
It has the disadvantages of evaluating one or the other of its arguments
a second time
(including
\indextext{side effects}%
side effects)
and generating more code than a function if invoked several times.
It also cannot have its address taken,
as it has none.

\begin{codeblock}
#define max(a, b) ((a) > (b) ? (a) : (b))
\end{codeblock}

The parentheses ensure that the arguments and
the resulting expression are bound properly.
\exitexample

\pnum
\enterexample
To illustrate the rules for redefinition and reexamination,
the sequence

\begin{codeblock}
#define x       3
#define f(a)    f(x * (a))
#undef  x
#define x       2
#define g       f
#define z       z[0]
#define h       g(@\tilde@
#define m(a)    a(w)
#define w       0,1
#define t(a)    a
#define p()     int
#define q(x)    x
#define r(x,y)  x ## y
#define str(x)  # x

f(y+1) + f(f(z)) % t(t(g)(0) + t)(1);
g(x+(3,4)-w) | h 5) & m
    (f)^m(m);
p() i[q()] = { q(1), r(2,3), r(4,), r(,5), r(,) };
char c[2][6] = { str(hello), str() };
\end{codeblock}

results in

\begin{codeblock}
f(2 * (y+1)) + f(2 * (f(2 * (z[0])))) % f(2 * (0)) + t(1);
f(2 * (2+(3,4)-0,1)) | f(2 * (@\tilde@ 5)) & f(2 * (0,1))^m(0,1);
int i[] = { 1, 23, 4, 5, };
char c[2][6] = { "hello", "" };
\end{codeblock}
\exitexample

\pnum
\enterexample
To illustrate the rules for creating character string literals
and concatenating tokens,
the sequence

\begin{codeblock}
#define str(s)      # s
#define xstr(s)     str(s)
#define debug(s, t) printf("x" # s "= %d, x" # t "= %s", @\textbackslash@
               x ## s, x ## t)
#define INCFILE(n)  vers ## n
#define glue(a, b)  a ## b
#define xglue(a, b) glue(a, b)
#define HIGHLOW     "hello"
#define LOW         LOW ", world"

debug(1, 2);
fputs(str(strncmp("abc@\textbackslash@0d", "abc", '@\textbackslash@4')  // this goes away
    == 0) str(: @\atsign\textbackslash@n), s);
#include xstr(INCFILE(2).h)
glue(HIGH, LOW);
xglue(HIGH, LOW)
\end{codeblock}

results in

\begin{codeblock}
printf("x" "1" "= %d, x" "2" "= %s", x1, x2);
fputs("strncmp(@\textbackslash@"abc@\textbackslash\textbackslash@0d@\textbackslash@", @\textbackslash@"abc@\textbackslash@", '@\textbackslash\textbackslash@4') == 0" ": @\atsign\textbackslash@n", s);
#include "vers2.h"    @\textit{(after macro replacement, before file access)}@
"hello";
"hello" ", world"
\end{codeblock}

or, after concatenation of the character string literals,

\begin{codeblock}
printf("x1= %d, x2= %s", x1, x2);
fputs("strncmp(@\textbackslash@"abc@\textbackslash\textbackslash@0d@\textbackslash@", @\textbackslash@"abc@\textbackslash@", '@\textbackslash\textbackslash@4') == 0: @\atsign\textbackslash@n", s);
#include "vers2.h"    @\textit{(after macro replacement, before file access)}@
"hello";
"hello, world"
\end{codeblock}

Space around the
\tcode{\#}
and
\tcode{\#\#}
tokens in the macro definition is optional.
\exitexample

\pnum
\enterexample
To illustrate the rules for placemarker preprocessing tokens, the sequence

\begin{codeblock}
#define t(x,y,z) x ## y ## z
int j[] = { t(1,2,3), t(,4,5), t(6,,7), t(8,9,),
  t(10,,), t(,11,), t(,,12), t(,,) };
\end{codeblock}

results in

\begin{codeblock}
int j[] = { 123, 45, 67, 89,
  10, 11, 12, };
\end{codeblock}
\exitexample

\pnum
\enterexample
To demonstrate the redefinition rules,
the following sequence is valid.

\begin{codeblock}
#define OBJ_LIKE      (1-1)
#define OBJ_LIKE      @\tcode{/* white space */ (1-1) /* other */}@
#define FUNC_LIKE(a)   ( a )
#define FUNC_LIKE( a )(     @\tcode{/* note the white space */ \textbackslash}@
                a @\tcode{/* other stuff on this line}@
                  @\tcode{*/}@ )
\end{codeblock}

But the following redefinitions are invalid:

\begin{codeblock}
#define OBJ_LIKE    (0)      // different token sequence
#define OBJ_LIKE    (1 - 1)  // different white space
#define FUNC_LIKE(b) ( a )   // different parameter usage
#define FUNC_LIKE(b) ( b )   // different parameter spelling
\end{codeblock}
\exitexample

\pnum
\enterexample
Finally, to show the variable argument list macro facilities:

\begin{codeblock}
#define debug(...) fprintf(stderr, @\mname{VA_ARGS}@)
#define showlist(...) puts(#@\mname{VA_ARGS}@)
#define report(test, ...) ((test) ? puts(#test) : printf(@\mname{VA_ARGS}@))
debug("Flag");
debug("X = %d\n", x);
showlist(The first, second, and third items.);
report(x>y, "x is %d but y is %d", x, y);
\end{codeblock}

results in

\begin{codeblock}
fprintf(stderr, "Flag");
fprintf(stderr, "X = %d\n", x);
puts("The first, second, and third items.");
((x>y) ? puts("x>y") : printf("x is %d but y is %d", x, y));
  
\end{codeblock}
\exitexample
\indextext{macro!replacement|)}

\rSec1[cpp.line]{Line control}%
\indextext{preprocessing directives!line control}%
\indextext{\idxcode{\#line}|see{preprocessing directives, line control}}

\pnum
The string literal of a
\tcode{\#line}
directive, if present,
shall be a character string literal.

\pnum
The
\term{line number}
of the current source line is one greater than
the number of new-line characters read or introduced
in translation phase 1~(\ref{lex.phases})
while processing the source file to the current token.

\pnum
A preprocessing directive of the form

\begin{ncsimplebnf}
\terminal{\# line} digit-sequence new-line
\end{ncsimplebnf}

causes the implementation to behave as if
the following sequence of source lines begins with a
source line that has a line number as specified
by the digit sequence (interpreted as a decimal integer).
If the digit sequence specifies zero
or a number greater than 2147483647,
the behavior is undefined.

\pnum
A preprocessing directive of the form

\begin{ncsimplebnf}
\terminal{\# line} digit-sequence \terminal{"} s-char-sequence\opt \terminal{"} new-line
\end{ncsimplebnf}

sets the presumed line number similarly and changes the
presumed name of the source file to be the contents
of the character string literal.

\pnum
A preprocessing directive of the form

\begin{ncsimplebnf}
\terminal{\# line} pp-tokens new-line
\end{ncsimplebnf}

(that does not match one of the two previous forms)
is permitted.
The preprocessing tokens after
\tcode{line}
on the directive are processed just as in normal text
(each identifier currently defined as a macro name is replaced by its
replacement list of preprocessing tokens).
If the directive resulting after all replacements does not match
one of the two previous forms, the behavior is undefined;
otherwise, the result is processed as appropriate.

\rSec1[cpp.error]{Error directive}%
\indextext{preprocessing directives!error}%
\indextext{\idxcode{\#error}|see{preprocessing directives, error}}

\pnum
A preprocessing directive of the form

\begin{ncsimplebnf}
\terminal{\# error} pp-tokens\opt new-line
\end{ncsimplebnf}

causes the implementation to produce
a diagnostic message that includes the specified sequence of preprocessing tokens,
and renders the program ill-formed.

\rSec1[cpp.pragma]{Pragma directive}%
\indextext{preprocessing directives!pragma}%
\indextext{\idxcode{\#pragma}|see{preprocessing directives, pragma}}

\pnum
A preprocessing directive of the form

\begin{ncsimplebnf}
\terminal{\# pragma} pp-tokens\opt new-line
\end{ncsimplebnf}

causes the implementation to behave
in an \impldef{\#pragma@\tcode{\#pragma}} manner.
The behavior might cause translation to fail or cause the translator or
the resulting program to behave in a non-conforming manner.
Any pragma that is not recognized by the implementation is ignored.

\rSec1[cpp.null]{Null directive}%
\indextext{preprocessing directives!null}

\pnum
A preprocessing directive of the form

\begin{ncsimplebnf}
\terminal{\#} new-line
\end{ncsimplebnf}

has no effect.

\rSec1[cpp.predefined]{Predefined macro names}
\indextext{macro!predefined}%
\indextext{name!predefined macro|see{macro, predefined}}

\pnum
The following macro names shall be defined by the implementation:

\begin{description}[style=nextline]

\indextext{\idxcode{\unun cplusplus}}%
\item[\tcode{\xname{cplusplus}}]
The name \tcode{\,\xname{cplusplus}} is defined
to the value
\tcode{\cppver}
when
compiling a \Cpp translation unit.\footnote{It is intended that future
versions of this standard will
replace the value of this macro with a greater value.
Non-conforming compilers should use a value with at most
five decimal digits.}

\indextext{\xname{DATE}@\mname{DATE}}%
\item[\mname{DATE}]
The date of translation of the source file:
a character string literal of the form
\tcode{"Mmm~dd~yyyy"},
where the names of the months are the same as those generated
by the
\tcode{asctime}
function,
and the first character of
\tcode{dd}
is a space character if the value is less than 10.
If the date of translation is not available,
an \impldef{text of \mname{DATE} when date of translation is not available} valid date
shall be supplied.

\indextext{\xname{FILE}@\mname{FILE}}%
\item[\mname{FILE}]
The presumed name of the current source file (a character string
literal).\footnote{The presumed source file name and line number can be changed
by the \tcode{\#line} directive.}

\indextext{\xname{LINE}@\mname{LINE}}%
\item[\mname{LINE}]
The presumed line number (within the current source file) of the current source line
(an integer literal).%
\addtocounter{footnote}{-1}\footnotemark\

\indextext{\xname{STDC_HOSTED}@\mname{STDC_HOSTED}}%
\indextext{\xname{STDC_HOSTED}@\mname{STDC_HOSTED}!implementation-defined}%
\item[\mname{STDC_HOSTED}]
The integer literal \tcode{1} if the implementation is a hosted
implementation or the integer literal \tcode{0} if it is not.

\indextext{\xname{TIME}@\mname{TIME}}%
\item[\mname{TIME}]
The time of translation of the source file:
a character string literal of the form
\tcode{"hh:mm:ss"}
as in the time generated by the
\tcode{asctime}
function.
If the time of translation is not available,
an \impldef{text of \mname{TIME} when time of translation is not available} valid time shall be supplied.
\end{description}

\pnum
The following macro names are conditionally defined by the implementation:

\begin{description}[style=nextline]
  \indextext{\xname{STDC}@\mname{STDC}}%
  \indextext{\xname{STDC}@\mname{STDC}!implementation-defined}%
\item[\mname{STDC}]
Whether \mname{STDC} is predefined and if so, what its value is,
are \impldef{definition and meaning of \mname{STDC}}.

\indextext{\xname{STDC_MB_MIGHT_NEQ_WC}@\mname{STDC_MB_MIGHT_NEQ_WC}}%
\indextext{\xname{STDC_MB_MIGHT_NEQ_WC}@\mname{STDC_MB_MIGHT_NEQ_WC}!implementation-defined}%
\item[\mname{STDC_MB_MIGHT_NEQ_WC}]
The integer literal \tcode{1}, intended to indicate that, in the encoding for
\tcode{wchar_t}, a member of the basic character set need not have a code value equal to
its value when used as the lone character in an ordinary character literal.

\indextext{\xname{STDC_VERSION}@\mname{STDC_VERSION}}%
\indextext{\xname{STDC_VERSION}@\mname{STDC_VERSION}!implementation-defined}%
\item[\mname{STDC_VERSION}]
Whether \mname{STDC_VERSION} is predefined and if so, what its value is,
are \impldef{definition and meaning of \mname{STDC_VERSION}}.

\indextext{\xname{STDC_ISO_10646}@\mname{STDC_ISO_10646}}%
\indextext{\xname{STDC_ISO_10646}@\mname{STDC_ISO_10646}!implementation-defined}%
\item[\mname{STDC_ISO_10646}]
An integer literal of the form \tcode{yyyymmL} (for example,
\tcode{199712L}).
If this symbol is defined, then every character in the Unicode required set, when
stored in an object of type \tcode{wchar_t}, has the same value as the short identifier
of that character. The \defn{Unicode required set} consists of all
the characters that are defined by ISO/IEC 10646, along with
all amendments and technical corrigenda as of the specified year and month.

\indextext{\xname{STDCPP_STRICT_POINTER_SAFETY}@\mname{STDCPP_STRICT_POINTER_SAFETY}}%
\indextext{\xname{STDCPP_STRICT_POINTER_SAFETY}@\mname{STDCPP_STRICT_POINTER_SAFETY}!implementation-defined}%
\item[\mname{STDCPP_STRICT_POINTER_SAFETY}]
Defined, and has the value integer literal 1, if and only if the implementation
has strict pointer safety~(\ref{basic.stc.dynamic.safety}).

\indextext{\xname{STDCPP_THREADS}@\mname{STDCPP_THREADS}}%
\indextext{\xname{STDCPP_THREADS}@\mname{STDCPP_THREADS}!implementation-defined}%
\item[\mname{STDCPP_THREADS}]
Defined, and has the value integer literal 1, if and only if a program
can have more than one thread of execution~(\ref{intro.multithread}).

\end{description}

\pnum
The values of the predefined macros
(except for
\mname{FILE}
and
\mname{LINE})
remain constant throughout the translation unit.

\pnum
If any of the pre-defined macro names in this subclause,
or the identifier
\tcode{defined},
is the subject of a
\tcode{\#define}
or a
\tcode{\#undef}
preprocessing directive,
the behavior is undefined.
Any other predefined macro names shall begin with a
leading underscore followed by an uppercase letter or a second
underscore.

\rSec1[cpp.pragma.op]{Pragma operator}%
\indextext{macro!pragma operator}%
\indextext{operator!pragma|see{macro, pragma operator}}

A unary operator expression of the form:

\begin{ncsimplebnf}
\terminal{_Pragma} \terminal{(} string-literal \terminal{)}
\end{ncsimplebnf}

is processed as follows: The string literal is \term{destringized}
by deleting the \tcode{L} prefix, if present, deleting the leading and trailing
double-quotes, replacing each escape sequence \tcode{\textbackslash"} by a double-quote, and
replacing each escape sequence \tcode{\textbackslash\textbackslash} by a single
backslash. The resulting sequence of characters is processed through translation phase 3
to produce preprocessing tokens that are executed as if they were the
\grammarterm{pp-tokens} in a pragma directive. The original four preprocessing
tokens in the unary operator expression are removed.

\enterexample

\begin{codeblock}
#pragma listing on "..\listing.dir"
\end{codeblock}

can also be expressed as:

\begin{codeblock}
_Pragma ( "listing on \"..\\listing.dir\"" )
\end{codeblock}

The latter form is processed in the same way whether it appears literally
as shown, or results from macro replacement, as in:

\begin{codeblock}
#define LISTING(x) PRAGMA(listing on #x)
#define PRAGMA(x) _Pragma(#x)

LISTING( ..\listing.dir )
\end{codeblock}

\exitexample%
\indextext{preprocessing directives|)}
