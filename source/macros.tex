%!TEX root = std.tex

% This is the draft-to-xml version of the macro.tex file.
% Use it (only) when compiling for LXir to convert the draft to XML.
% Only the core language is supported at the moment;
% front and back matter, appendices, library parts etc. are not supported.
% dyp, 2014-04

% dyp:
% I added the package enumitem in the harmonize branch to replace the manual
% technique used to add newlines in description lists in [cpp.predefined]
% ex.
%
% \begin{description}
% \item[\tcode{\xname{cplusplus}}]\\
% The name \tcode{\,\xname{cplusplus}} is defined
% ...
%
% now:
%
% \begin{description}
% \item[\tcode{\xname{cplusplus}}]
% The name \tcode{\,\xname{cplusplus}} is defined
% ...
%
% However, enumitem seems not to be supported (natively) by lxir.
% Here's a hack that provides the minimal support required at this point.
\let\descriptionorig=\description
\let\enddescriptionorig=\enddescription
\renewenvironment{description}[1][]
{\descriptionorig}
{\enddescriptionorig}


% Definitions and redefinitions of special commands

%%--------------------------------------------------
%% Difference markups
\definecolor{addclr}{rgb}{0,.6,.6}
\definecolor{remclr}{rgb}{1,0,0}
\definecolor{noteclr}{rgb}{0,0,1}

% dyp: these macros are not used in recent drafts and therefore currently not supported

%\renewcommand{\added}[1]{\textcolor{addclr}{\uline{#1}}}
%\newcommand{\removed}[1]{\textcolor{remclr}{\sout{#1}}}
%\renewcommand{\changed}[2]{\removed{#1}\added{#2}}
%
%\newcommand{\nbc}[1]{[#1]\ }
%\newcommand{\addednb}[2]{\added{\nbc{#1}#2}}
%\newcommand{\removednb}[2]{\removed{\nbc{#1}#2}}
%\newcommand{\changednb}[3]{\removednb{#1}{#2}\added{#3}}
%\newcommand{\remitem}[1]{\item\removed{#1}}
%
%\newcommand{\ednote}[1]{\textcolor{noteclr}{[Editor's note: #1] }}
%% \newcommand{\ednote}[1]{}
%
%\newenvironment{addedblock}
%{
%\color{addclr}
%}
%{
%\color{black}
%}
%\newenvironment{removedblock}
%{
%\color{remclr}
%}
%{
%\color{black}
%}

%%--------------------------------------------------
%% Sectioning macros.  
% Each section has a depth, an automatically generated section 
% number, a name, and a short tag.  The depth is an integer in 
% the range [0,5].  (If it proves necessary, it wouldn't take much
% programming to raise the limit from 5 to something larger.)

\makeatletter

% dyp: remove the "Chapter" from "Chapter X" headings
\def\@chapapp{}

% dyp: remove the \quad from \@seccntformat
\def\@seccntformat#1{\csname the#1\endcsname}

% dyp: \paragraph and \subparagraph don't produce sectionHeaders
% this seems to be a non-trivial problem. First, the call stack:
%   \paragraph in latex/base/book.cls
%   \@startsection in latex kernel
%   \@sect in lxir/lxir.sty
% and here's the problem:
%   - book.cls defines a negative AFTERSKIP size (the 5th argument to \@startsection)
%   - \@sect directly emits lxir tags if AFTERSKIP > 0,
%     but only defines \@svsechd otherwise (containing the tags)
%   - \@sect calls \@xsect (defined in the latex kernel),
%     which *should* output the lxir tags for the first \par (using \everypar)
%   - this does not seem to work: \everypar seems not be be invoked
%   - no tags are emitted :(
% instead of meddling with LXir here, I chose to redefine the AFTERSKIP to a positive value
\renewcommand{\paragraph}{\@startsection{paragraph}{4}{\z@}%
                                        {3.25ex \@plus1ex \@minus.2ex}%
                                        {1ex}%
                                        {\normalfont\normalsize\bfseries}}
\renewcommand{\subparagraph}{\@startsection{paragraph}{5}{\parindent}%
                                           {3.25ex \@plus1ex \@minus.2ex}%
                                           {1ex}%
                                           {\normalfont\normalsize\bfseries}}

\makeatother


% The basic sectioning command.  Example:
%    \Sec1[intro.scope]{Scope}
% defines a first-level section whose name is "Scope" and whose short
% tag is intro.scope.  The square brackets are mandatory.
\def\Sec#1[#2]#3{%
\ifcase#1\let\s=\chapter
      \or\let\s=\section
      \or\let\s=\subsection
      \or\let\s=\subsubsection
      \or\let\s=\paragraph
      \or\let\s=\subparagraph
      \fi%
% dyp: moved the label for easier restructuring of XML
%\s[#3]{#3\hfill[#2]}\label{#2}}
\s[#3]{#3\label{#2}}}

% A convenience feature (mostly for the convenience of the Project
% Editor, to make it easy to move around large blocks of text):
% the \rSec macro is just like the \Sec macro, except that depths 
% relative to a global variable, SectionDepthBase.  So, for example,
% if SectionDepthBase is 1,
%   \rSec1[temp.arg.type]{Template type arguments}
% is equivalent to
%   \Sec2[temp.arg.type]{Template type arguments}
\newcounter{SectionDepthBase}
\newcounter{scratch}

\def\rSec#1[#2]#3{%
\setcounter{scratch}{#1}
\addtocounter{scratch}{\value{SectionDepthBase}}
\Sec{\arabic{scratch}}[#2]{#3}}

\newcommand{\synopsis}[1]{\textbf{#1}}

%%--------------------------------------------------
% Indexing

% dyp: disabling indices for now
\renewcommand{\makeindex}[1][dummy]{}
\renewcommand{\index}[2][dummy]{}

% locations
\newcommand{\indextext}[1]{\index[generalindex]{#1}}
\newcommand{\indexlibrary}[1]{\index[libraryindex]{#1}}
\newcommand{\indexgram}[1]{\index[grammarindex]{#1}}
\newcommand{\indeximpldef}[1]{\index[impldefindex]{#1}}

\newcommand{\indexdefn}[1]{\indextext{#1}}
\newcommand{\indexgrammar}[1]{\indextext{#1}\indexgram{#1}}
\newcommand{\impldef}[1]{\indeximpldef{#1}implementation-defined}

% appearance
\newcommand{\idxcode}[1]{#1@\tcode{#1}}
\newcommand{\idxhdr}[1]{#1@\tcode{<#1>}}
\newcommand{\idxgram}[1]{#1@\textit{#1}}

%%--------------------------------------------------
% General code style
\newcommand{\CodeStyle}{\ttfamily}
\newcommand{\CodeStylex}[1]{\texttt{#1}}

% Code and definitions embedded in text.
%\newcommand{\tcode}[1]{\CodeStylex{#1}}
\newcommand{\tcode}[1]{\xBegin{tcode}#1\xEnd{tcode}}% dyp: used for inline code
%\newcommand{\techterm}[1]{\textit{#1}\xspace}
\newcommand{\techterm}[1]{\xBegin{techterm}#1\xEnd{tcode}\xspace}% dyp: not used?
%\newcommand{\defnx}[2]{\indexdefn{#2}\textit{#1}\xspace}
\newcommand{\defnx}[2]{\indexdefn{#2}\xBegin{defnx}#1\xEnd{defnx}\xspace}% dyp: indexed definition of a term
%\newcommand{\defn}[1]{\defnx{#1}{#1}}
\newcommand{\defn}[1]{\defnx{#1}{#1}}% dyp: ??
%\newcommand{\term}[1]{\textit{#1}\xspace}
\newcommand{\term}[1]{\xBegin{term}#1\xEnd{term}\xspace}% dyp: many different uses


% dyp: \grammarterm is sometimes used "with two arguments", even though the macro only takes a single one
%      ex. \grammarterm{}{postfix-expression}
%      this effectively disables the formatting of the term
%      I've re-enables the formatting in the harmonize branch, but provide additional support here
%      for both usages
%\newcommand{\grammarterm}[1]{\textit{#1}\xspace}
% dyp: (usage note for \grammarterm) inline occurence of a grammar term
\def\grammartermTwo#1#2{\xBegin{grammarterm}#1\xEnd{grammarterm}#2\xspace}
\newcommand{\grammartermOne}[1]{\xBegin{grammarterm}#1\xEnd{grammarterm}\xspace}
% from http://tex.stackexchange.com/a/10405 ; thanks to Joseph Wright
\NewDocumentCommand\grammarterm{m+g}{%
  \IfNoValueTF{#2}
    {\grammartermOne{#1}}
    {\grammartermTwo{#1}{#2}}%
}

%\newcommand{\placeholder}[1]{\textit{#1}\xspace}
\newcommand{\placeholder}[1]{% dyp: placeholder variables, such as foo and bar
  \xBegin{placeholder}#1\xEnd{placeholder}\xspace%
}

% For comparison of multiple ICS in [over.match.best]
%\newcommand{\ICS}[2]{$\text{ICS}#1(\text{\tcode{#2}})$}
\newcommand{\ICS}[2]{\xEmptyAA{ICS}{index=#1}{arg=#2}}

%%--------------------------------------------------
%% allow line break if needed for justification
\newcommand{\brk}{\discretionary{}{}{}}
%  especially for scope qualifier
\newcommand{\colcol}{\brk::\brk}

%%--------------------------------------------------
%% Macros for funky text
%\newcommand{\Rplus}{\protect\hspace{-.1em}\protect\raisebox{.35ex}{\smaller{\smaller\textbf{+}}}}% dyp: fancy plus for nice rendering of "C++" (see below)
% \newcommand{\Rplus}{+}
%\newcommand{\Cpp}{\mbox{C\Rplus\Rplus}\xspace}
\newcommand{\Cpp}{\xEmpty{cpp}\xspace}
%\newcommand{\CppIII}{\Cpp 2003\xspace}
\newcommand{\CppIII}{\xEmpty{cpp2003}\xspace}
%\newcommand{\CppXI}{\Cpp 2011\xspace}
\newcommand{\CppXI}{\xEmpty{cpp2011}\xspace}
%\newcommand{\opt}{{\ensuremath{_\mathit{opt}}}\xspace}
\newcommand{\opt}{\xEmpty{opt}\xspace}
\newcommand{\shl}{<{<}}
\newcommand{\shr}{>{>}}
\newcommand{\dcr}{-{-}}
\newcommand{\exor}{\^{}}
\newcommand{\bigoh}[1]{\ensuremath{\mathscr{O}(#1)}}
% \renewcommand{\tilde}{{\smaller\ensuremath{\sim}}}    % extra level of braces is necessary
%\renewcommand{\tilde}{\protect\raisebox{-.5ex}{\larger\textasciitilde}}
\renewcommand{\tilde}{\textasciitilde}
\def\~{\tilde}

%%--------------------------------------------------
%% States and operators
\newcommand{\state}[2]{\tcode{#1}\ensuremath{_{#2}}}
\newcommand{\bitand}{\ensuremath{\, \mathsf{bitand} \,}}
\newcommand{\bitor}{\ensuremath{\, \mathsf{bitor} \,}}
\newcommand{\xor}{\ensuremath{\, \mathsf{xor} \,}}
\newcommand{\rightshift}{\ensuremath{\, \mathsf{rshift} \,}}
\newcommand{\leftshift}[1]{\ensuremath{\, \mathsf{lshift}_#1 \,}}

%% Notes and examples
%\newcommand{\EnterBlock}[1]{[\,\textit{#1:}\xspace}
%\newcommand{\ExitBlock}[1]{\textit{\,---\,end #1}\,]\xspace}
%\newcommand{\enternote}{\EnterBlock{Note}}
\newcommand{\enternote}{\xBegin{note}}
%\newcommand{\exitnote}{\ExitBlock{note}}
\newcommand{\exitnote}{\xEnd{note}}
%\newcommand{\enterexample}{\EnterBlock{Example}}
\newcommand{\enterexample}{\xBegin{example}}
%\newcommand{\exitexample}{\ExitBlock{example}}
\newcommand{\exitexample}{\xEnd{example}}

% dyp: library currently not supported

%% Library function descriptions
%\newcommand{\Fundescx}[1]{\textit{#1}\xspace}
%\newcommand{\Fundesc}[1]{\Fundescx{#1:}}
%\newcommand{\required}{\Fundesc{Required behavior}}
%\newcommand{\requires}{\Fundesc{Requires}}
%\newcommand{\effects}{\Fundesc{Effects}}
%\newcommand{\postconditions}{\Fundesc{Postconditions}}
%\newcommand{\postcondition}{\Fundesc{Postcondition}}
%\newcommand{\preconditions}{\requires}
%\newcommand{\precondition}{\requires}
%\newcommand{\returns}{\Fundesc{Returns}}
%\newcommand{\throws}{\Fundesc{Throws}}
%\newcommand{\default}{\Fundesc{Default behavior}}
%\newcommand{\complexity}{\Fundesc{Complexity}}
%\newcommand{\remark}{\Fundesc{Remark}}
%\newcommand{\remarks}{\Fundesc{Remarks}}
%\newcommand{\note}{\remark}
%\newcommand{\notes}{\remarks}
%\newcommand{\realnote}{\Fundesc{Note}}
%\newcommand{\realnotes}{\Fundesc{Notes}}
%\newcommand{\errors}{\Fundesc{Error conditions}}
%\newcommand{\sync}{\Fundesc{Synchronization}}
%\newcommand{\implimits}{\Fundesc{Implementation limits}}
%\newcommand{\replaceable}{\Fundesc{Replaceable}}
%\newcommand{\exceptionsafety}{\Fundesc{Exception safety}}
%\newcommand{\returntype}{\Fundesc{Return type}}
%\newcommand{\cvalue}{\Fundesc{Value}}
%\newcommand{\ctype}{\Fundesc{Type}}
%\newcommand{\ctypes}{\Fundesc{Types}}
%\newcommand{\dtype}{\Fundesc{Default type}}
%\newcommand{\ctemplate}{\Fundesc{Class template}}
%\newcommand{\templalias}{\Fundesc{Alias template}}
%
%%% Cross reference
%\newcommand{\xref}{\textsc{See also:}\xspace}
%\newcommand{\xsee}{\textsc{See:}\xspace}

%% NTBS, etc.
% dyp: these are macros to print the abbreviations of
%      * null-terminated byte string
%      * null-terminated multi-byte string
%      and so on
%      in small capitals
%\newcommand{\NTS}[1]{\textsc{#1}\xspace}
\newcommand{\NTS}[1]{\xBegin{NTS}#1\xEnd{NTS}\xspace}%{\textsc{#1}\xspace}
\newcommand{\ntbs}{\NTS{ntbs}}
\newcommand{\ntmbs}{\NTS{ntmbs}}
\newcommand{\ntwcs}{\NTS{ntwcs}}
\newcommand{\ntcxvis}{\NTS{ntc16s}}
\newcommand{\ntcxxxiis}{\NTS{ntc32s}}

%% Code annotations
\newcommand{\EXPO}[1]{\textit{#1}}
%\newcommand{\expos}{\EXPO{exposition only}}
\newcommand{\expos}{\xEmptyA{exposition-only}{category=code}}
%\newcommand{\impdef}{\EXPO{implementation-defined}}
\newcommand{\impdef}{\xEmptyA{implementation-defined}{category=code}}% dyp: e.g. the return type of std::bind
%\newcommand{\impdefx}[1]{\indeximpldef{#1}\EXPO{implementation-defined}}
\newcommand{\impdefx}[1]{\indeximpldef{#1}\impdef}
%\newcommand{\notdef}{\EXPO{not defined}}
\newcommand{\notdef}{\xEmptyA{notdef}{category=code}}% dyp: currently unused

%\newcommand{\UNSP}[1]{\textit{\texttt{#1}}}% dyp: too generic formatting, using more specific replacements
%\newcommand{\unspec}{\UNSP{unspecified}\xspace}
\newcommand{\unspec}{\xEmptyA{unspecified}{category=code}}
%\newcommand{\unspecbool}{\UNSP{unspecified-bool-type}}
\newcommand{\unspecbool}{\xEmptyA{unspecified-bool-type}{category=code}}
%\newcommand{\seebelow}{\UNSP{see below}}
\newcommand{\seebelow}{\xEmptyA{see-below}{category=code}}
%\newcommand{\unspecuniqtype}{\UNSP{unspecified unique type}}
\newcommand{\unspecuniqtype}{\xEmptyA{unspecified-unique-type}{category=code}}
%\newcommand{\unspecalloctype}{\UNSP{unspecified allocator type}}
\newcommand{\unspecalloctype}{\xEmptyA{unspecified-allocator-type}{category=code}}

%% Double underscore
% dyp: multiple underscores seem to have been separated with spaces for formatting reasons
%      they should actually be adjacent (w/o space characters)
%\newcommand{\unun}{\_\,\_}
\newcommand{\unun}{\_\_}
%\newcommand{\xname}[1]{\unun\,#1}
\newcommand{\xname}[1]{\unun#1}
%\newcommand{\mname}[1]{\tcode{\unun\,#1\,\unun}}
\newcommand{\mname}[1]{\tcode{\unun#1\unun}}

%% Ranges
\newcommand{\Range}[4]{\tcode{#1\brk{}#3,\brk{}#4\brk{}#2}\xspace}
\newcommand{\crange}[2]{\Range{[}{]}{#1}{#2}}
\newcommand{\brange}[2]{\Range{(}{]}{#1}{#2}}
\newcommand{\orange}[2]{\Range{(}{)}{#1}{#2}}
\newcommand{\range}[2]{\Range{[}{)}{#1}{#2}}

%% Change descriptions
% dyp: currently unused (?)
%\newcommand{\diffdef}[1]{\hfill\break\textbf{#1:}\xspace}
%\newcommand{\change}{\diffdef{Change}}
%\newcommand{\rationale}{\diffdef{Rationale}}
%\newcommand{\effect}{\diffdef{Effect on original feature}}
%\newcommand{\difficulty}{\diffdef{Difficulty of converting}}
%\newcommand{\howwide}{\diffdef{How widely used}}

%% Miscellaneous
% dyp: uniquens means "unique namespace"
%\newcommand{\uniquens}{\textrm{\textit{\textbf{unique}}} }
\newcommand{\uniquens}{\xEmpty{uniquens}}
%\newcommand{\stage}[1]{\item{\textbf{Stage #1:}}\xspace}% dyp: currently not supported
%\newcommand{\doccite}[1]{\textit{#1}\xspace}
\newcommand{\doccite}[1]{\xBegin{doccite}#1\xEnd{doccite}\xspace}
%\newcommand{\cvqual}[1]{\textit{#1}}
\newcommand{\cvqual}[1]{\xBegin{cvqual}#1\xEnd{cvqual}}
\newcommand{\cv}{\cvqual{cv}}
%\renewcommand{\emph}[1]{\textit{#1}\xspace}
\renewcommand{\emph}[1]{\xBegin{emph}#1\xEnd{emph}\xspace}
%\newcommand{\numconst}[1]{\textsl{#1}\xspace}
\newcommand{\numconst}[1]{\xBegin{numconst}#1\xEnd{numconst}}
%\newcommand{\logop}[1]{{\footnotesize #1}\xspace}
\newcommand{\logop}[1]{\xBegin{logop}#1\xEnd{logop}}

%%--------------------------------------------------
%% Environments for code listings.

% We use the 'listings' package, with some small customizations.  The
% most interesting customization: all TeX commands are available
% within comments.  Comments are set in italics, keywords and strings
% don't get special treatment.
\newcommand{\foo}{\xEmpty{newline}}
\lstset{language=C++,
        basicstyle=\small\CodeStyle,
        keywordstyle=,
        stringstyle=,
        xleftmargin=1em,
        showstringspaces=false,
        commentstyle=\itshape\rmfamily,
        columns=flexible,
        keepspaces=true,
        texcl=true}

% Our usual abbreviation for 'listings'.  Comments are in 
% italics.  Arbitrary TeX commands can be used if they're 
% surrounded by @ signs.
% dyp: I hacked the lst environment similarly to how LXir hacked the verbatim environment:
% overwriting the macros used in the implementation to inject certain tags
% and later replacing them in xslt
\makeatletter
  \let\lstoutputspaceorig=\lst@outputspace
  \def\lst@outputspace{\xEmpty{verbatimSpace}\lstoutputspaceorig}
  
  \let\lstNewLineorig=\lst@NewLine
  \def\lst@NewLine{\xEmpty{verbatimLineBreak}\lstNewLineorig}
\makeatother

\newcommand{\CodeBlockSetup}{
 \lstset{escapechar=@}
%\renewcommand{\tcode}[1]{\textup{\CodeStylex{##1}}}
 \renewcommand{\tcode}[1]{\xBegin{tcode}##1\xEnd{tcode}}
%\renewcommand{\techterm}[1]{\textit{\CodeStylex{##1}}}
 \renewcommand{\techterm}[1]{\xBegin{techterm}##1\xEnd{techterm}}
%\renewcommand{\term}[1]{\textit{##1}}
 \renewcommand{\term}[1]{\xBegin{term}##1\xEnd{term}}
%\renewcommand{\grammarterm}[1]{\textit{##1}}
 \renewcommand{\grammarterm}[1]{\xBegin{grammarterm}##1\xEnd{grammarterm}}
}
\lstnewenvironment{codeblock}{\CodeBlockSetup\xBegin{codeblock}}{\xEnd{codeblock}}

% A code block in which single-quotes are digit separators
% rather than character literals.
\lstnewenvironment{codeblockdigitsep}{
 \CodeBlockSetup
 \xBegin{codeblock}
 \lstset{deletestring=[b]{'}}
}{\xEnd{codeblock}}

% Permit use of '@' inside codeblock blocks (don't ask)
% dyp: "don't ask" means that '@' is used as the listing's escape character,
%      i.e. to be able to insert latex code
%      not sure why '$' has not been used.
\makeatletter
\newcommand{\atsign}{@}
\makeatother

%%--------------------------------------------------
%% Indented text
\newenvironment{indented}
%{\list{}{}\item\relax}
%{\endlist}
{\xBegin{indented}}
{\xEnd{indented}}

% dyp: library not supported (yet)
%
%%%--------------------------------------------------
%%% Library item descriptions
%\lstnewenvironment{itemdecl}
%{
% \lstset{escapechar=@,
% xleftmargin=0em,
% aboveskip=2ex,
% belowskip=0ex	% leave this alone: it keeps these things out of the
%				% footnote area
% }
%}
%{
%}
%
%\newenvironment{itemdescr}
%{
% \begin{indented}}
%{
% \end{indented}
%}


%%--------------------------------------------------
%% Bnf environments

% dyp:
% bnf probably stands for Backus-Naur Form
% (well it's the environment used to describe the grammar rules)
% The bnf environment itself describes the rule for ONE nonterminal.
% If multiple nonterminals shall be defined, multiple bnf environments are used (contiguously).
% Here's an example from [declarations]
%
% \begin{bnf}
% \nontermdef{declaration-seq}\br
%     declaration\br
%     declaration-seq declaration
% \end{bnf}%
%
% As you can see, the first element is always a \nontermdef (or \nonterm), which
% * indexes a nonterminal (\nontermdef)
% * introduces a special formatting (both).
% Then, multiple sequences follow, each preceded by an \br.
% All but the first \br can be interpreted as an OR ('|' in Backus-Naur notation).
% Since I think it's awful to introduce the semantics solely via this linebreak notation,
% I'll hack the \br and add an XSLT to get a proper list.
% (Note: Although a \list is used here, it only serves indentation purposes, as far as I can see.)
%
% .... p.s. the \br sometimes is also used for pure formatting, see [dcl.enum]


%\newlength{\BnfIndent}
%\setlength{\BnfIndent}{\leftmargini}
%\newlength{\BnfInc}
%\setlength{\BnfInc}{\BnfIndent}
%\newlength{\BnfRest}
%\setlength{\BnfRest}{2\BnfIndent}
%\newcommand{\BnfNontermshape}{\small\rmfamily\itshape}
%\newcommand{\BnfTermshape}{\small\ttfamily\upshape}
%\newcommand{\nonterminal}[1]{{\BnfNontermshape #1}}
\newcommand{\nonterminal}[1]{\xBegin{nonterminal}#1\xEnd{nonterminal}}%{{\BnfNontermshape #1}}


\newenvironment{bnfbase}
 {
%\newcommand{\nontermdef}[1]{\nonterminal{##1}\indexgrammar{\idxgram{##1}}:}
 \newcommand{\nontermdef}[1]{\indexgrammar{\idxgram{##1}}\xBegin{nontermdef}##1\xEnd{nontermdef}}% dyp: removed a trailing ':'
%\newcommand{\terminal}[1]{{\BnfTermshape ##1}\xspace}
 \newcommand{\terminal}[1]{\xBegin{terminal}##1\xEnd{terminal}\xspace}
%\newcommand{\descr}[1]{\normalfont{##1}}
 \newcommand{\descr}[1]{\xBegin{description}##1\xEnd{description}}
%\newcommand{\bnfindentfirst}{\BnfIndent}
 \newcommand{\bnfindentfirst}{}
%\newcommand{\bnfindentinc}{\BnfInc}
 \newcommand{\bnfindentinc}{\leftmargini}
%\newcommand{\bnfindentrest}{\BnfRest}
%\begin{minipage}{.9\hsize}
%\newcommand{\br}{\hfill\\}
 \newcommand{\br}{\xEmpty{grammar-rule}}
 \renewcommand{\textnormal}[1]{\xBegin{description}##1\xEnd{description}}
 }
 {
 %\end{minipage}
 }

% dyp: I have no need for this, I directly hacked bnfkeywordtab instead
%\newenvironment{BnfTabBase}[1]
%{
% \begin{bnfbase}
% #1
% \begin{indented}
% \begin{tabbing}
% \hspace*{\bnfindentfirst}\=\hspace{\bnfindentinc}\=\hspace{.6in}\=\hspace{.6in}\=\hspace{.6in}\=\hspace{.6in}\=\hspace{.6in}\=\hspace{.6in}\=\hspace{.6in}\=\hspace{.6in}\=\hspace{.6in}\=\hspace{.6in}\=\kill}
%{
% \end{tabbing}
% \end{indented}
% \end{bnfbase}
%}

% dyp:
% This environment is used twice (in [lex.operators]/1 and [over.oper]/1).
% It defines a nonterminal, where the rules are simple terminals and there are so many of them
% that you cannot (practically) put them on individual lines.
% In LaTeX code, the rules are separated by \> and the line breaks defined via \br
% 
% Example:
% 
% operator: one of
%     new  delete  new  delete[]
%     +  -  *  /  %  ^  &  |  ~
%     [...]
%
% note the spacing between the keywords, produced by a \> (a \medmuskip)
\newenvironment{bnfkeywordtab}
{
%\begin{BnfTabBase}{\BnfTermshape}
 \begin{bnfbase}
 \xBegin{bnfkeywordtab}
 \renewcommand{\>}{\xEmpty{next-keyword}}
 \renewcommand{\br}{\xEmpty{br-hint}}
}
{
%\end{BnfTabBase}
 \xEnd{bnfkeywordtab}
 \end{bnfbase}
}

% dyp: This environment is used differently than the bnfkeywordtab, despite the similar name.
% It defines a nonterminal; the rules are manually indented.
% Allows additional indentation within the rule [cpp] and line-breaks within a rule,
% the following line is then continued with one more level of indentation [lex.ccon].
% ex. from [gram.lex]
% \begin{bnftab}
% \nontermdef{c-char}\br
% \>\textnormal{any member of the source character set except}\br
% \>\>\textnormal{the single-quote \terminal{'}, [...]}
% \>escape-sequence\br
% \>universal-character-name
% \end{bnftab}
\newenvironment{bnftab}
{
%\begin{BnfTabBase}{\BnfNontermshape}
 \begin{bnfbase}
 \xBegin{bnftab}
 \renewcommand{\br}{\xEmpty{br}}
 \renewcommand\>{\xEmpty{indent}}
}
{
 %\end{BnfTabBase}
 \xEnd{bnftab}
 \end{bnfbase}
}

% dyp: the simple bnf looks just like a list of rules, what they define is specified in the text
%      surrounding the simplebnf environment
% Only occurs as ncsimplebnf.
% Does not contain manual indenting and does not define a nonterminal inside the environment.
\newenvironment{simplebnf}
{
 \begin{bnfbase}
%\BnfNontermshape
%\begin{indented}
 \xBegin{simplebnf}
 \renewcommand{\\}{\xEmpty{grammar-rule}}
}
{
%\end{indented}
 \xEnd{simplebnf}
 \end{bnfbase}
}


% dyp:
% Always contains a \nontermdef.
% Defines the nonterminal using one or more rules (automatic indenting).
% One exception is [dcl.fct.def.general]/1, where "function-body:" has been used
% as the nonterminal which is to be defined, instead of \nontermdef{function-body}
% or \nonterminal{function-body:}
% The non-copied version usually doesn't contain a nonterminal definition;
% the single exception being [over.match.call]/3
\newenvironment{bnf}
{
 \begin{bnfbase}
%\list{}
%   {
%   \setlength{\leftmargin}{\bnfindentrest}
%   \setlength{\listparindent}{-\bnfindentinc}
%   \setlength{\itemindent}{\listparindent}
%   }
%\BnfNontermshape
%\item\relax
 \xBegin{bnf}
}
{
%\endlist
 \xEnd{bnf}
 \end{bnfbase}
}

% non-copied versions of bnf environments
\newenvironment{ncbnftab}
{
 \begin{bnftab}
}
{
 \end{bnftab}
}

\newenvironment{ncsimplebnf}
{
 \begin{simplebnf}
}
{
 \end{simplebnf}
}

\newenvironment{ncbnf}
{
 \begin{bnf}
}
{
 \end{bnf}
}

%%--------------------------------------------------
%% Drawing environment
%
% usage: \begin{drawing}{UNITLENGTH}{WIDTH}{HEIGHT}{CAPTION}
\newenvironment{drawing}[4]
{
\newcommand{\mycaption}{#4}
\begin{figure}[h]
\setlength{\unitlength}{#1}
\begin{center}
\begin{picture}(#2,#3)\thicklines
}
{
\end{picture}
\end{center}
\caption{\mycaption}
\end{figure}
}

%%--------------------------------------------------
%% Environment for imported graphics
% usage: \begin{importgraphic}{CAPTION}{TAG}{FILE}

% dyp: graphics are not yet supported
%      the current macros should allow adding this support easily

\newenvironment{importgraphic}[3]
{%
%\newcommand{\cptn}{#1}
%\newcommand{\lbl}{#2}
%\begin{figure}[htp]\centering%
%\includegraphics[scale=.35]{#3}
\begin{figure}[htp]%
\xBegin{importgraphic}#3\xEnd{importgraphic}%
\caption{#1}\label{#2}
}
{
%\caption{\cptn}\label{\lbl}%
\end{figure}}

%% enumeration display overrides
% enumerate with lowercase letters
\newenvironment{enumeratea}
{
 \renewcommand{\labelenumi}{\alph{enumi})}
 \begin{enumerate}
}
{
 \end{enumerate}
}

% enumerate with arabic numbers
\newenvironment{enumeraten}
{
 \renewcommand{\labelenumi}{\arabic{enumi})}
 \begin{enumerate}
}
{
 \end{enumerate}
}

%%--------------------------------------------------
%% Definitions section
% usage: \definition{name}{xref}
%\newcommand{\definition}[2]{\rSec2[#2]{#1}}
% for ISO format, use:
% dyp: definitions only occur in [intro.defs] and are a hybrid of subsections and list items
\newcommand{\definition}[2]
 {\hfill\vspace{.25ex plus .5ex minus .2ex}\\
 \addtocounter{subsection}{1}%
%\textbf{\thesubsection\hfill\relax[#2]}\\
%\textbf{#1}
 \xBegin{definition}\xAttr{position=\thesubsection}\xAttr{level=2}#1\xEnd{definition}
 \label{#2}\\
 }
